\section{Implementing Similarity Rater Using LCS}
The \textbf{longest common subsequence}
\footnote{\href{https://en.wikipedia.org/wiki/Longest_common_subsequence_problem}
{https://en.wikipedia.org/wiki/Longest_common_subsequence_problem}} 
(LCS) problem is the 
problem of finding the longest subsequence common 
to all sequences in a set of sequences 
(often just two sequences). It differs 
from the longest common substring problem: 
unlike substrings, subsequences are not 
required to occupy consecutive positions within 
the original sequences.
For example, consider the sequences (ABCD) 
and (ACBAD). They have 5 length-2 common 
subsequences: (AB), (AC), (AD), (BD), and (CD); 
2 length-3 common subsequences: (ABD) and (ACD); 
and no longer common subsequences. So (ABD) and (ACD) 
are their longest common subsequences.


Let two sequences be defined as follows: 
$ X=(x_{1}x_{2}\cdots x_{m})$
$ X=(x_{1}x_{2}\cdots x_{m})$ and 
$ Y=(y_{1}y_{2}\cdots y_{n})$
$ Y=(y_{1}y_{2}\cdots y_{n})$.
The prefixes of $X$ are 
$X_{1,2,\dots ,m}$
$ X_{1,2,\dots ,m}$; the prefixes of 
$Y$ are $ Y_{1,2,\dots ,n}$
$ Y_{1,2,\dots ,n}$. Let 
$ {\mathit {LCS}}(X_{i},Y_{j})$
$ {\mathit {LCS}}(X_{i},Y_{j})$
represent the set of longest common subsequence of 
prefixes $ X_{i}$ and $ Y_{j}$. 
This set of sequences is given by the following.
\begin{equation}
\label{eq:lcs}
LCS(X_i,Y_j)=
\begin{cases}
\emptyset & \text{if } i=0 \text{ or } j =0\\
LCS(x_{i-1}, y_{j-1})^\frown x_i & \text{if } i,j>0 \text{ and } x_i = y_j\\
max\{LCS(X_i, Y_{j-1}), LCS(X_{i-1}, Y_j)\} & \text{if } i,j>0 \text{ and } x_i \neq y_j
\end{cases}
\end{equation}


\subsection{Python Implementatoin}
Using the LCS algorithm we can find the longest common 
sequence in the two or more strings. But applying this 
method to long documents can be irrelevant. Because in document 
similarity problems we mostly care how semantically those are the 
same and finding a common sequence of characters has little to do 
with that. Instead, we can split the documents into words and find the 
longest common sequence of words.

Now, let’s implement this in Python. The LCS function is going to 
take two lists of strings and find LCS among those. Using a dynamic 
programming approach and with use of tabulation we solve the LCS 
problem.

As an example, let’s  say, we have two following lists of strings. 
$$x = [‘Hey’, ‘Mehran’, ‘how’, ‘are’, ‘you’, ‘doing’, ‘today’] $$
$$ y = [‘Hey’, ‘Mehrdad’, ‘how’, ‘are’, ‘doing?’] $$

Using recursive formula in the equation \ref{eq:lcs} and taking a 
bottom up approach we can fill a table similar to what shown in the 
\ref{tab:table1} and find the LCS by returning the bottom right cell
in the table.
\begin{table}[H]
  \begin{center}
    \caption{LCS of two lists. Blue colored list is the answer.}
    \label{tab:table1}
    \resizebox{\columnwidth}{!}{%
    \begin{tabular}{|c|c|c|c|c|c|c|c|c|}
      \toprule % <-- Toprule here
      & $\emptyset$ & \textbf{Hey} & \textbf{Mehran,} & \textbf{how} 
      & \textbf{are} & \textbf{you} & \textbf{doning} 
      & \textbf{today?}\\
      \midrule % <-- Midrule here
      $\emptyset$ & $[]$ & $[]$ & $[]$ & $[]$ & $[]$ & $[]$ & $[]$ &$[]$  \\
      \midrule
      \textbf{Hey} & $[]$ & \textcolor{darkgreen}{$['hey']$} &  $ \leftarrow ['hey']$ & $ \leftarrow ['hey']$ & $\leftarrow ['hey']$ & $\leftarrow ['hey']$ & $\leftarrow ['hey']$ & $ \leftarrow['hey']$ \\
      \midrule % <-- Midrule here
      \textbf{Mehrdad,} & $[]$ &$\uparrow ['hey']$ & $\uparrow ['hey']$ & $\uparrow ['hey']$ & $\uparrow ['hey']$ & $\uparrow ['hey']$ & $\uparrow ['hey']$ & $\uparrow ['hey']$ \\
      \midrule % <-- Midrule here
      \textbf{how} & $[]$ & $\uparrow ['hey']$ & $\uparrow ['hey']$ & $\textcolor{darkgreen}{['hey', 'how']}$ & $\leftarrow ['hey', 'how']$ & $\leftarrow ['hey', 'how']$ & $\leftarrow ['hey','how']$ & $\leftarrow ['hey','how']$ \\
      \midrule % <-- Midrule here
      \textbf{are} & $[]$ & $\uparrow ['hey']$ & $\uparrow ['hey']$ & $\uparrow ['hey', 'how']$ & $\textcolor{darkgreen}{['hey', 'how', 'are']}$ & $\leftarrow ['hey', 'how', 'are']$ & $\leftarrow['hey','how','are]$ & $\leftarrow['hey','how', 'are']$ \\
      \midrule % <-- Midrule here
      \textbf{doing?} & $[]$ & $\uparrow ['hey']$ & $\uparrow ['hey']$ & $\uparrow ['hey', 'how']$ & $\uparrow ['hey', 'how', 'are']$ & $\uparrow ['hey', 'how', 'are']$ & $\uparrow ['hey','how','are]$ & \textcolor{blue}{$\uparrow  ['hey','how', 'are']$} \\ 
      \bottomrule % <-- Bottomrule here
    \end{tabular}%
    }
  \end{center}
\end{table}

The code implemented in Python is shown in Listing \ref{lis:lcs}. First, the table is filled with empty
lists. Looping through the table, strting with second row and second column
( first row and first column initialaized with empty before)
using similar formula as equation \ref{eq:lcs} fill the table with approporiate values.
\textit{indx} function is used to map each index of texts with right index in
table. Because the talble have one extra row and column to find the right index for the table we need
to add 1 to the index of each character in the text. For convenience a \textit{lambda}
function is written for that.
\pythonexternal[label=lis:lcs, caption=Implementatoin of LCS in python]{codes/lcs.py}
\subsubsection{Runtime and Memory Complexity}
The runtime complexity of function is shown in Listing \ref{lis:lcs} is $O(m \times n)$, because
it needs to loop through the table of size $m \times n$.
There is subtlty in calculating Memory Complexity. In Python, lists
only have refrences to their elements and don't store whole elements.
And in total there is $ m \times n$ refrences in the table with number of
LCS distinct strings refrences. That in the worst case gives $O(m \times n \times min\{n,m\})$.